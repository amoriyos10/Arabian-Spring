% ---------------------------------*- Latex -*---------------------------------
% Filename: descriptionSolver.tex
% Description: 
% Author: Fabian Wermelinger
% Email: fabianw@student.ethz.ch
% Created: Tue Dec 13 20:40:11 2011 (+0100)
% Version: 
% Last-Updated: Tue Dec 13 22:16:35 2011 (+0100)
%           By: Fabian Wermelinger
%     Update #: 30
% -----------------------------------------------------------------------------
% descriptionSolver.tex starts here
% -----------------------------------------------------------------------------

\subsection{Solving Process}
\label{sec:descriptioSolving}

Once a network has been defined, the relation between the nodes is fully
determined.  If $\mathcal{N}$ is the set of all nodes contained in the network,
then each node $i$ has a set of neighbors $\mathcal{S}_i \subset \mathcal{N}$.
The neighbors $j\in\mathcal{S}_i$ of node $i$ may be \emph{direct} or
\emph{indirect} neighbors.  A direct neighbor $j$ is one that is linked
adjacent to node $i$, where an indirect neighbor node is one that is a neighbor
of a neighbor in a successive fashion.  Each node is characterized by two main
features, \ie, a state and a threshold.
\begin{definition}
  The state of a node is a binary variable.  The value $0$ means that the state
  for a situation is \emph{pro}, whereas the value $1$ is \emph{contra} the
  situation.  The state is subject to change.
\end{definition}
\begin{definition}
  The threshold of a node is a numeric value on the interval $[0,1]$.  A
  threshold of $0$ means that the node will remain in its state no matter what
  the external influences are.  A threshold of $1$ acts opposite.  That is, a
  node with a threshold of $1$ will \emph{always} change its state from pro to
  contra, no matter what the external influences are.  If it already was in
  contra state, it will remain in that state.  Hence, a node with a threshold
  of $1$ acts as a ``seed'' node to initiate some event.  The threshold is a
  constant.
\end{definition}

With the above definitions, the solving process is as follows:
\begin{enumerate}
\item Loop over time
\item Generate a sequential update list
\item Loop over the sequential update list
\item Calculate the neighbor state residual and compare it with the threshold
  of the actual node
\item If the condition applies, update the state of the node
\end{enumerate}

\subsubsection{The Neighbor State Residual}
\label{sec:nbrStateRes}

The neighbor state residual is a scalar which is a measure of an overall state
for all neighbor nodes $j\in\mathcal{S}_i$.  It is used to decide whether a
nodal state is changed by comparing it with the nodal threshold.  If there are
$n$ elements in $\mathcal{S}_i$, then let $\bm{r}_i\in A^n$ be a vector of
dimension $n$, where its elements $r_j$ hold the state of each node
$j\in\mathcal{S}_i$.  The set $A$ holds the two elements $0$ and $1$, which
represent the state.  The neighbor state residual, $\rho_i$, is then defined to
be the Euclidean norm of $\bm{r}_i$, \ie,
\begin{align}
  \label{eq:nbrStateRes}
  \rho_i &= \norm{\bm{r}_i}
\end{align}

So if $\rho_i$ is greater or equal to the threshold of node $i$, then the state
will be forced to contra, \ie, to the value $1$.  The calculation then proceeds
for all nodes in the sequential update list before the next time step is
processed.

%%% Local Variables: 
%%% mode: latex
%%% TeX-master: "master"
%%% End:

% -----------------------------------------------------------------------------
% descriptionSolver.tex ends here
% -----------------------------------------------------------------------------

