%------------ Latex -----------
%describes the implementation and function of agents.m

\subsection{Implementation of the Agents function}
\label{sec:ImplementAgents }

The complete \matlab code can be found in Appendix ~\ref{sec:agents.m}. The
agents function fulfils two purposes. The first is to set and determine the
properties every single node (agent) in the network has. As second it acts as
interface between the generation of the network and the solver.

The function consists of a for loop that iterates through all nodes of the
network and generates a list containing all the agents, with there
properties. These properties are the state, the threshold, a list of all the
neighbours and which country or cluster the agent is part of. The state tells
whether the agent already joined the revolution (1) or not (0), therefore all
agents initially get the value 0. The neighbour property is a list of all the
agents the particular agent has a connection to in the network.

The threshold is the percentage of surrounding agents that sill have to have
state 0, so the agent does not join the opposition. It gets determined by a
uniformly distributed random number between 0 and 1. Dependent on the given
parameters a certain percentage can get a fixed threshold. Furthermore a few
agents get a threshold of 1. Those are the agents, that start the
revolution. How many of these rebels are placed in the network and how far
they are apart depends on the input parameters.