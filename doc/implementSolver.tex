% ---------------------------------*- Latex -*---------------------------------
% Filename: implementSolver.tex
% Description: 
% Author: Fabian Wermelinger
% Email: fabianw@student.ethz.ch
% Created: Tue Dec 13 22:16:47 2011 (+0100)
% Version: 
% Last-Updated: Tue Dec 13 23:18:21 2011 (+0100)
%           By: Fabian Wermelinger
%     Update #: 9
% -----------------------------------------------------------------------------
% implementSolver.tex starts here
% -----------------------------------------------------------------------------

\subsection{Implementation of the Solver Function}
\label{sec:implementSolver}

The full code of the solver function can be found in
Appendix~\ref{sec:solverSIRv3.m}.  This section describes the implementation of
Section~\ref{sec:descriptioSolving} in \matlab code.  Additionally, the actual
solver implemented in \matlab also supports a variant of a SIR (Susceptible
Infected Recovered) model for a cellular automaton.

The solver consists mainly of two nested \textsl{for-loops}, where the outer
loop is over a time vector and the inner loop is over a sequential list of
agents, which are to be updated in the active time-step.  In each of these
loops a call to a subroutine may be executed.  Since the subroutines are solver
specific functions, they are appended at the bottom of the solver function.
Therefore, they are only visible to the solver.  The subroutines are
sufficiently commented in the code and will not be discussed here in more
detail.  However, in order to aid the understanding of how the solver works in
general, the statements in the nested loops need further explanation.


%%% Local Variables: 
%%% mode: latex
%%% TeX-master: "master"
%%% End:

% -----------------------------------------------------------------------------
% implementSolver.tex ends here
% -----------------------------------------------------------------------------
