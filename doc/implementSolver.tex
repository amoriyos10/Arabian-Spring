% ---------------------------------*- Latex -*---------------------------------
% Filename: implementSolver.tex
% Description: 
% Author: Fabian Wermelinger
% Email: fabianw@student.ethz.ch
% Created: Tue Dec 13 22:16:47 2011 (+0100)
% Version: 
% Last-Updated: Thu Dec 15 21:13:07 2011 (+0100)
%           By: Fabian Wermelinger
%     Update #: 24
% -----------------------------------------------------------------------------
% implementSolver.tex starts here
% -----------------------------------------------------------------------------

\subsection{Implementation of the Solver Function}
\label{sec:implementSolver}

The full code of the solver function can be found in
Appendix~\ref{sec:solverSIRv3.m}.  This section describes the implementation of
Section~\ref{sec:descriptioSolving} in \matlab code.  Additionally, the actual
solver implemented in \matlab also supports a variant of a SIR (Susceptible
Infected Recovered) model for a cellular automaton.

The solver consists mainly of two nested \textsl{for}-loops, where the outer
loop is over a time vector and the inner loop is over a sequential list of
agents, which are to be updated in the active time-step.  In each of these
loops a call to a subroutine may be executed.  Since the subroutines are solver
specific functions, they are appended at the bottom of the solver function.
Therefore, they are only visible to the solver.  The subroutines are
sufficiently commented in the code and will not be discussed here in more
detail.  However, in order to aid the understanding of how the solver works in
general, the statements in the nested loops need further explanation.

The main algorithm starts at line~33 and ends on line~75 in the source code
found in Appendix~\ref{sec:solverSIRv3.m}.  The very first thing that is done
in every time-step is the calculation of the individual cluster residuals, as
well as the global residual for the whole network.  If the whole network
consists of only one cluster, the two residuals are the same.  The calculated
residuals provide a measure of the development of the social behavior over
time.  If it is close to zero, the majority of people are content with the
current situation.  If it is close to one, the opposite is true.  On line~39
the sequential agent update list is calculated.  It is a uniformly distributed
list of random agent indices that get update in the current time-step.  The
list is of random length (also uniformly distributed) but at most
\texttt{maxUpdatedAgents} long.  The constant \texttt{maxUpdatedAgents} is
calculated on line~26 and is some percentage (defined in the parameter
structure) of $\mathcal{N}$, thus, if the set $\mathcal{U}$ contains the
sequential update list of the current time-step, then
$\mathcal{U}\subseteq\mathcal{N}$.

On line~40, a number of noisy agents is calculated in the same way as it is
descried above for the \texttt{maxUpdatedAgents} constant.  The only difference
is that \texttt{nNoize} is a percentage of the set $\mathcal{U}$.  The
\textsl{if}-block from line~42--51 then adds the noize and removes the noisy
agents from the sequential update list.  The update algorithm starts with the
\textsl{for}-loop on line~53.  The algorithm first checks whether the agent is
subject to update.  That is, if its state is zero and if $\rho_i$ is strong
engough to pass the agents threshold.  Then the infection phase of the SIR
implementation takes effect.  If $\beta = 1$, the infection is not applied at
all.  If all of these tests are true, then the agent's state is updated to
contra (\ie, ``$1$'') on line~59.  Note that $\rho_i$ is computed with a
recursive function in order to allow an arbitrary depth of indirect neighbors.
If \texttt{nbrDepth} is set to unity in the parameter list, then only direct
neighbors are taken into account.

The \textsl{if}-block from line~66 to~69 applies the removal phase of the SIR
implementation.  Note that if $\gamma = 0$, the expression is never true and
the removal phase is ignored.  Therfore, if one sets the parameter $\beta = 1$
and $\gamma = 0$, then the SIR model is turned off.  The last \textsl{if}-block
is a write operation, which generates a \texttt{.csv} file.  The file is simply
the list of agents with contra state at the current time-step.  We have used
this file to generate visualizations of time dependent developments in the
network.  We have used R\footnote{R is a collection of software packages for
  statistical computing and graphic generation.  See
  \url{http://www.r-project.org/} for more information.} to generate the
network plots.

\subsubsection{Limitations}
\label{sec:solverLimitations}

The current implementation of the solver function is limited by means that it
only allows state updates in one direction.  That is, an agent's state can only
be changed from pro to contra.  For further development, one may consider to
allow the reverse as well.  However, this implies that if the initial agents
all have a pro state and at least one agent acts as a seed, then noize must be
added in order to allow an agent to change its state in the reverse direction.

The current implementation also assumes that each agent knows all of its direct
and indirect neighbors.  Such an assumption is conservative, as it might be
very well true that an agent only knows some of its neighbors, this is
especially true for the indirect neighbors.  Hence, an additional parameter
could be introduced that defines such behavior.  This would have immediate
effect on $\rho_i$ and eventually influences the decision making of the agent.
Also, note that if the number of indirect neighbors is increased (by increasing
the \texttt{nbrDepth} parameter), the computation time of the solver is also
increased (depending on the network, it may increase drastically).

%%% Local Variables: 
%%% mode: latex
%%% TeX-master: "master"
%%% End:

% -----------------------------------------------------------------------------
% implementSolver.tex ends here
% -----------------------------------------------------------------------------

