% ---------------------------------*- Latex -*---------------------------------
% Filename: introAndMotivation.tex
% Description: 
% Author: Fabian Wermelinger
% Email: fabianw@student.ethz.ch
% Created: Sat Dec 10 14:45:04 2011 (+0100)
% Version: 
% Last-Updated: Thu Dec 15 20:37:49 2011 (+0100)
%           By: Fabian Wermelinger
%     Update #: 15
% -----------------------------------------------------------------------------
% introAndMotivation.tex starts here
% -----------------------------------------------------------------------------

\section{Introduction and Motivation}
\label{sec:introAndMotivation}

A few months ago in some Arabian countries there were some kind of
revolutionary movements. In some countries they were successful while in other
countries they struggled. Our aim of this research project is to understand the
mechanisms of the success of the Arabian Spring movements.  We want to build a
society based on a small world graph with, e.g. three clusters, where each
cluster represents a country adjacent to another. These clusters shall be
formed with individual probabilities, generating different homogeneity of each
country as such. With the focus on one country, we want to analyze the
spreading of opinion changes between nodes (where each node has an agent
assigned to) in that particular cluster. Further, the propagation of the
opinion formation across the clusters shall be investigated.

Previous research showed that behavioral diffusion in social networks is
different from, \eg, the spread of an infectious desease
\cite{centola2010spread}.  In social networks, an individual usually requires
multiple contact with another individual(s) in order to change its behavior.
Also, the depth and speed of the behavioral diffusion is larger in small world
networks than randomly generated networks \cite{centola2010spread}.  The main
content of this work is the development of a model to simulate the behavioral
diffusion for events such as the Arabian Spring \cite{timeLineArabSpring}.  The
network is composed of three sub-networks, each representing a country.  One
question to follow is to find out the influence on the whole system if these
countries were constructed as small world networks or random networks.  In one
of these countries a ``seed'' is placed to initiate an imbalance in the
society's behavior.  The information then spreads across the network depending
on the behavior of its neighbors (and may be also its higher order neighbors,
\ie, neighbors of neighbors and so on).  The diffusion across countries is
realized through the link between two individuals from different countries who
know each other.  However, the threshold of willingness to change behavior may
be different for an individual of the country which is the root of the
behavioral change than that of an individual of a neighbor country, since it is
only affected indirectly.

Once a model has been developed, the development of the behavioral diffusion of
the model may be compared to the results of the experiment conducted in
\cite{centola2010spread}, which is of similar type as the implemented model.

%%% Local Variables: 
%%% mode: latex
%%% TeX-master: "master"
%%% End: 

% -----------------------------------------------------------------------------
% introAndMotivation.tex ends here
% -----------------------------------------------------------------------------
